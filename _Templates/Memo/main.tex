%\title{Overleaf Memo Template}
% Using the texMemo package by Rob Oakes
\documentclass[a4paper,11pt]{texMemo}
\usepackage[english]{babel}
\usepackage{graphicx, xcolor}

\memoto{All Staff}
\memofrom{Connor McHugh}
\memosubject{Refreshed ISI Stock Management Strategy}
\memodate{\today}
\logo{\includegraphics[width=0.3\textwidth]{MonextraNew.png}}

\begin{document}
\maketitle

Going forward we will be implementing a new strategy for Instant Scratch-It stock management. We will be replacing the current system where we sign ISI books out as we withdraw them from the safe to the drawer with a once a month reconciliation process.
\\
\\
Following are the key changes under this new system:
\begin{itemize}
    \item When ISI stock is received, you will no longer need to bundle and date the books, but when putting new stock into the safe, try to bundle games of the same type as much as possible the make reconciliation easier.
    \item Books can be taken out of the safe in any quantity, you no longer need to worry about taking out only full bundles.
    \item Once you have taken books from the safe, they only need to be written into the ISI count book, then they can go straight into the drawer.
    \item There will be a reconciliation process that must be performed on the last day of every month, where the total number of books in the drawer and safe is checked against a report from the Lotteries Terminal.
\end{itemize}

It will now be more important than ever that when moving ISI books from the safe to the drawer, you ensure that you have written the correct quantities into the ISI count book, and have had another staff member double check both the quantities, and the math. \color{red}\textbf{There will be no way to double check at the end of the day.} \color{black}
\\
\par
The attached document outlines in detail the entire ISI Stock Management Strategy.

\end{document}